\documentclass{article}

% Package Pygments for fancy typesetting of code
\usepackage{fancyvrb}

\usepackage[T1]{fontenc}
\usepackage[utf8]{inputenc}
\usepackage{color}

% Hyperlinks in PDF: \href{url}{linktext}
\definecolor{linkcolor}{rgb}{0,0,0.4}
\usepackage[%
    colorlinks=true,
    linkcolor=linkcolor,
    urlcolor=linkcolor,
    citecolor=black,
    filecolor=black,
]{hyperref}

\begin{document}

\title{INF3331 - Oblig1}

\author{Gøran Frost}
\date{\today}

% Generate title, author, and date
\maketitle

\section{Bash}

\label{sec:calc}

\subsection{find\_new\_files.sh}
\label{sec:add}
\begin{quote}
``List opp alle filer som er modifisert de siste n dagene. La utskriften inkludere og være sortet på filstørrelse.''
\end{quote}
\begin{Verbatim}
./find_new_files.sh dir n
\end{Verbatim}
Scriptet find\_new\_files.sh leter igjennom \emph{dir} etter filer (\emph{find} som har blitt endret de siste \emph{n} dager ( \emph{ctime}, og så for hver slik fil skriver ut diskforbruket (\emph{du} til filen og filnavnet.

\label{sec:add:eq}
\subsection{find\_word.sh}
\begin{quote}
``Find alle filer som inneholder et gitt ord. Merk at vi snakker om på innsiden av
filen, ikke filnavnet.''
\end{quote}
\begin{Verbatim}
./find_word.sh dir word
\end{Verbatim}
Scriptet leter igjennom \emph{dir} etter alle filer og mapper (\emph{find}). Så går vi igjennom alle resultatene, og om de er filer sjekker vi om de inneholder \emph{word} ved hjelp av \emph{grep}. Hvis filen inneholder \emph{word}, skriver vi ut filnavnet. 

\subsection{sized\_delete.sh}
\begin{quote}
``Slett alle filer i filtreet med størrelse større enn en gitt verdi. Størrelsen er gitt
i kilobyte. Print ut navnene på filene som slettes.''
\end{quote}
\begin{Verbatim}
./sized_delete.sh dir size
\end{Verbatim}
Her går vi også igjennom alle filene i \emph{dir} (\emph{find -type f}, og sjekker enkelt ved hjelp
 av \emph{stat} størrelsen på filen. Dette sammenlignes med $\emph{size}*1000$ 
(for å gjøre om fra KB), og så sletter vi filen om den er større enn \emph{size} KB.  
\subsection{sort\_file.sh}
\label{sec:py}
\begin{quote}
``Sorter linjene i en fil og lagre dem i en ny fil.''
\end{quote}
\begin{Verbatim}
./sort_file file1 file2
\end{Verbatim}
Siden vi allerede har et program som sorterer linjer i filer (\emph{sort}), sender vi 
bare filnavnene til denne.
\section{Python}

\subsection{random\_string()}
\begin{Verbatim}
random_string(length, prefix, legal_chars)

parametere 
  length : int
    lengde på returnert streng, minus prefix. default: 6
  prefix : string
    prefix for strengen. default = ""
  legal_chars : string
    streng av tilatte tegn. default = "[\w]"

returnerer
  res : string
\end{Verbatim}

\subsection{generate\_tree()}
\begin{Verbatim}
generate_tree(target, dirs, rec_depth, verbose)

parametere 
  target : string
    path til rotkatalogen til treet
  dirs : int
    maksimum antall underkataloger per katalog
  rec_depth : int
    maks dybde fra rotkatalogen
  verbose : bool
    hvis True, print alle kataloger som opprettes
\end{Verbatim}


\subsection{populate\_tree()}
\begin{Verbatim}
populate_tree(target, files, size, start_time, end_time, verbose)

parametere 
  target : string
    path til katalogen som filer skal lages i
  files : int
    maks antall filer i denne katalogen
  size : int
    maks størrelse på hver fil, i KB
  start_time : int
    minste mulige verdi for atime / mtime på filen (unix-format)
  end_time : int
    største mulige verdi for atime / mtime på filen (unix-format)
  verbose : bool
    hvis True, print alle filer som opprettes
\end{Verbatim}

\subsection{Argument Parsing}
argparse-modulen brukes for å håndtere alle argumenter.  
\end{document}
